\documentclass[UTF8]{beamer}
\usefonttheme[onlymath]{serif}
\setbeamertemplate{navigation symbols}{}
\usetheme{CambridgeUS}
%\setbeamerfont{text}{family*={ptm}, shape=\upshape, series=\mdseries}

\title{An Introduction to Polkadot}
\author{Wang Kaixuan}
\institute[SJTU]{Shanghai Jiao Tong University}

\usepackage{subfigure}

\include{Commands}

\begin{document}
	\usebeamerfont{text}
	\begin{frame}
		\titlepage
	\end{frame}

	\begin{frame}{Introduction}
		\begin{block}{Polkadot}
			Polkadot utilizes a central chain called the relay chain which communicates with multiple heterogeneous and independent sharded chains called parachains.\\
			I
		\end{block}
		\begin{block}{}
			content...
		\end{block}
	\end{frame}

\end{document} 