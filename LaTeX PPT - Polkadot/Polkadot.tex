\documentclass[UTF8]{beamer}
\usefonttheme[onlymath]{serif}
\setbeamertemplate{navigation symbols}{}
\usetheme{CambridgeUS}
%\setbeamerfont{text}{family*={ptm}, shape=\upshape, series=\mdseries}

\title{An Introduction to Polkadot}
\author{Wang Kaixuan}
\institute[SJTU]{Shanghai Jiao Tong University}

\usepackage{subfigure}

\include{Commands}

\begin{document}
	\usebeamerfont{text}
	\begin{frame}
		\titlepage
	\end{frame}

	\begin{frame}{Introduction}
		\begin{block}{Polkadot}
			Polkadot utilizes a central chain called the relay chain which communicates with multiple heterogeneous and independent sharded chains called parachains.\\
			I
		\end{block}
	\end{frame}

	\begin{frame}{Substrate}
		\begin{block}{What is Substrate?}
			Substrate is an open source, modular, and extensible framework for building  blockchains.
		\end{block}
		\begin{block}{}
			Substrate provides the core components of a blockchain: 
			\begin{enumerate}
				\item Database Layer
				\item Networking Layer
				\item Transaction Queue
				\item Consensus Engine
				\item Framework for Runtime Development
			\end{enumerate}
		Each of Which can be customized and extended
		\end{block}
	\end{frame}

	\begin{frame}{The Substrate Runtime}
		\begin{columns}
			\column{0.5\textwidth}
			\begin{block}{Runtime}
				\begin{enumerate}
					\item The runtime is the \textbf{block execution logic} of the blockchain, a.k.a the State Transition Function.
					\item It is composed of FRAME Pallets
				\end{enumerate}
			\end{block}
			\bigskip
			\bigskip
			\bigskip
			\bigskip
			\column{0.5\textwidth}
			\begin{figure}
				\centering
				\includegraphics[width=6cm]{figure/Runtime}
				\centering
				\includegraphics[width=6cm]{figure/Runtime2}
				\caption{Runtime and Frame}
			\end{figure}
		\end{columns}
	\end{frame}

	\begin{frame}{Building Blocks of Substrate}
		\begin{figure}
			\centering
			\includegraphics[width=0.7\linewidth]{"figure/Building Blocks of substrate"}
			\caption{}
			\label{fig:building-blocks-of-substrate}
		\end{figure}
	\end{frame}

	\begin{frame}{Why WebAssembly}
		\begin{block}{}
		Wasm is a platform independent executable format
		\begin{itemize}
			\item Wasm is Sandboxed
			\item Wasm is Fast
			\item Wasm is Compact
			\item Wasm is Well Supported
		\end{itemize}
	\end{block}
	\end{frame}
\end{document} 